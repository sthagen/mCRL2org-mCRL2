\documentclass{article}
\usepackage{fullpage}
\usepackage{tikz}
\usepackage{ stmaryrd }
\usepackage{algpseudocode}

\usetikzlibrary{automata, positioning, arrows}

\usepackage{amsfonts}
\newcommand{\pijl}[1]{\stackrel{#1}{\longrightarrow}}
\newcommand{\notpijl}[1]{\stackrel{#1}{\not\longrightarrow}}
\newcommand{\sem}[1]{\llbracket #1 \rrbracket}

\title{Notes on the bisimulation partitioner}
% Original by Jan Friso, Jan added information on distinguishing formulas.
\author{Jan Friso Groote and Jan Martens}
\begin{document}
\section{Introduction}
This document describes the algorithm for branching bisimulation reduction,
including the algorithm that is used to generate counterexamples.

\section{A partitioner for branching bisimulation}
The partitioner for branching bisimulation calculates 
whether states are bisimilar, branching bisimilar and stuttering
preserving branching bisimilar. It gets a state space and divides it into
a number of non-intersecting subsets of states called blocks. All
states in a block are bisimilar and have the same block index. 
Using this index it is straightforward to calculate whether two
states are equivalent (they have the same index) or to construct
the state space modulo this equivalence.

The algorithm works exactly as described in \cite{GV90}. As a preprocessing
step for (divergence
preserving) branching bisimulation all states that are strongly connected
via internal transitions are replaced by a single state. In case of divergence
preserving branching bisimulation this state gets a tau loop. In case of ordinary
branching bisimulation, there will not be a tau loop.

Then the algorithm for branching bisimulation is started. The state space is
partitioned into blocks. Initially, all states are put in one block. Repeatedly,
a block is split in two blocks until the partitioning has become stable. For
details see \cite{GV90}.

Furthermore, there is an option to obtain counter formulas for two non bisimilar
state. The algorithm for this is inspired by \cite{C90,K91}.

Given two non bisimilar states $s,t\in S$, where $S$ is the set of all states. A
{\it distinguishing formula} is a formula $\phi$ in Hennessy-Milner logic such
that $s\models\phi$ and $t\not\models\phi$. For branching bisimulation there is
always a distinguishing formula in the Hennessy-Milner logic extended with the
regular $\langle \tau^* \rangle$, and using the regular formula $\langle \tau +
\mathit{nil} \rangle \phi := \langle\tau\rangle \phi \vee \phi $. 

Following \cite{M23, M24} we implement the computation of minimal depth
distinguishing formulas. In the implementation two things can be noted different
from the references with the theoretical background.

\paragraph*{Filtering}
\begin{figure}[b]
  \begin{center}
  \begin{tikzpicture}[node distance=2.5cm]
      % States
      \node[state]   (s)                      {$s$};
      \node[state]   (s1)  [below of=s] {$s_1$};
      \node[state]   (t)  [right= 5cm of s] {$t$};
      \node[state]   (t1)  [below of=t] {$t_1$};
      \node[state]   (t2)  [ right of=t1] {$t_2$};
      % Transitions
      \path[->]
        (s)  edge node[left] {$a$} (s1)
        (t) edge node[left] {$a$} (t1)
        (t) edge node[right] {$a$} (t2)
        (s1) edge[loop below] node[below] {$a,b$} (s1)
        (t2) edge [loop below] node [below] {$a$} (t2);
  \end{tikzpicture}
  \end{center}
  \caption{Two LTSs with initial state $s$ and $t$ respectively.\label{fig:filtering}}
  \end{figure}

There is one post-processing step we call backwards filtering. It deals with the
following scenario. 

Consider the transition systems depicted in Figure~\ref{fig:filtering}. We
consider the scenario of computing a distinguishing formula $\phi(s,t)$ for the
states $s$ and $t$. By the partitioning algorithm we know that the transition
$s\pijl{a}s_1$ is a distinguishing observation. The algorithm has to recursively
find the distinguishing formula $\phi'$ such that $s_1 \in \sem{\phi'}$ and
$t_1,t_2 \not\in \sem{\phi'}$. This way $s\in\sem{\langle a\rangle \phi'}$ and
$t\not\in\sem{\langle a \rangle \phi'}$.

The algorithm starts by recursively computing a distinguishing formula for $s_1$
and $t_1$. It might compute $\phi(s_1, t_1) = \langle a \rangle \mathrm{true}$.
Since $t_2 \in \sem{ \langle a \rangle \mathrm{true}}$, it also computes the
distinguishing formula $\phi(s_1,t_2) = \langle b\rangle \mathrm{true}$. This
results in the formula $\phi(s,t) = \langle a \rangle (\langle a \rangle
\mathrm{true} \wedge \langle b \rangle \mathrm{true})$. We see that the addition
of the conjunct $\phi(s_1, t_2) = \langle b \rangle \mathrm{true}$ made the
first conjunct $\phi(s_1, t_1)$ obsolete (since $t_1\not\in\sem{\langle b\rangle
\mathrm{true}}$). This scenario is very hard to prevent a priori.  

To prevent a distinguishing formula with unnecessary conjuncts each conjunct is
reconsidered in a FIFO style. We compute the semantics of all other conjuncts
and see if it still achieves the goal. If this is the case, we can safely remove
that specific conjunct.

\paragraph*{Minimal depth partitioning for branching bisimulation}
According to~\cite{M24} we compute the partitions conform the correct $k$-depth
relations. Given the partition $\pi_i$ on level $i$ we want to construct
$\pi_{i{+}1}$ such that it stable w.r.t. the following signatures.

\[
  \mathit{sig}(s, \pi_i) := \{(B, a, B') \mid s \pijl{\tau^*} s' \pijl{a} s''
  , \textrm{where } s'\in B\in \pi_i, s''\in B'\in \pi_i\textrm{ and } (B \neq B' \vee a \neq \tau) \}  
\]

Performing this partitioning step efficiently is not obvious. We implemented the
following algorithm. This algorithm only works since we removed strongly
connected $\tau$ components in the preprocessing, hence there are no $\tau$
cycles.

\begin{algorithmic}[1]
  \State $\textit{frontier} := \{s \mid s\notpijl{\tau}\}$\;
  \State $\textit{done} := \emptyset$\;
  \While{$\textit{frontier} \neq \emptyset$}
  \State $s := \textit{frontier.pop\_front}()$\;
  \State $\textit{signature} := \{(B, a, B') \mid s\pijl{a} s' \textrm{ s.t. } s\in B, s'\in B',\textrm{ and }  B, B'\in \pi_i \textrm{ and } (a \neq \tau \vee B \neq B')\}$\;
  \ForAll{$s\pijl{\tau} s'$}
    \State $\textit{signature} := \textit{signature} \cup sigs[s']$\;
  \EndFor 
  \State $\textit{sigs}[s] := signature$\;
  \State $\textit{done} := \textit{done} \cup \{s\}$\:
  \ForAll{$s_p \pijl{\tau} s$} 
    \If{$\{s' \mid s_p \pijl{\tau} s'\} \setminus \textit{done} = \emptyset$}
      \State $\textit{frontier.push\_back}(s_p)$\;
    \EndIf
  \EndFor
  

  \EndWhile
\end{algorithmic}  



\begin{thebibliography}{10}
\bibitem{C90}
R.~Cleaveland. On automatically explaining bisimulation inequivalence.
In E.M.~Clarke and R.P.~Kurshan, editors, {\it Computer Aided Verification (CAV'90)}, volume
531 of {\it Lecture Notes in Computer Science}, Springer-Verlag, pages 364--372, 1990.

\bibitem{GV90}
J.F.~Groote and F.W.~Vaandrager. An efficient algorithm for branching bisimulation and stuttering equivalence. 
In M.S. Paterson, editor, Proceedings 17th ICALP, Warwick, volume 443 of Lecture Notes in Computer Science, 
pages 626-638. Springer-Verlag, 1990.

\bibitem{K91}
H.~Korver. Computing distinguishing formulas for branching bisimulation.
In K.G.~Larsen and A.~Skou, editors, {\it Computer Aided Verification (CAV'91)}, volume 575 of
{\it Lecture Notes in Computer Science}, Springer-Verlag, pages 13--23, 1991.

\bibitem{M23}
J.J.M.~Martens and J.F.~Groote. Computing Minimal Distinguishing Hennessy-Milner Formulas is NP-Hard, but Variants are Tractable.
In G.-A.~P'erez and J.-F. Raskin, editors, {\it CONCUR '23}, Volume 279 of
{\it LIPIcs}, Schloss Dagstuhl –- Leibniz-Zentrum für Informatik, pages 32:1--32:17, 2023.

\bibitem {M24}
J.J.M.~Martens and J.F.~Groote. Minimal Depth Distinguishing Formulas without Until for Branching Bisimulation. 
to appear in LNCS 2024

\end{thebibliography}
\end{document}
